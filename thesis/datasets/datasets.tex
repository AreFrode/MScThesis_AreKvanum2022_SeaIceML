\documentclass[../main/thesis.tex]{subfiles}

% Overleaf graphics path
\graphicspath{{thesis/datasets/figures/}}

% local graphics path
% \graphicspath{{/home/arefk/uio/MScThesis_AreKvanum2022_SeaIceML/thesis/datasets/figures/}}


\begin{document}
\section{Datasets}
\label{sec:datasets}
To facilitate the development and verification of a high resolution short-term deep learning sea ice forecasting system, several datasets from observations and physical model forecasting systems have been chosen. When selecting appropriate datasets, their spatial resolution as well as release frequency has been considered. Even though several observational sea ice concentration products which cover the region of interest exist, a lot of the satellite products based on passive microwave retrievals are of a too coarse resolution (e.g. \citet{Lavergne2019} or \citet{Kern2019}) to be able to aid in short term decision making \citep{Wagner2020}. Although Synthetic Aperture Radar (SAR) observations such as Sentinel 1A Interferometric Wide swath ($5\text{m} \times 20\text{m})$ or Extra-Wide swath ($20\text{m} \times 40\text{m}$) are on a sea ice structure resolving spatial resolution, the daily SAR coverage is sparse in the Arctic (See Supporting Figure \ref{fig:S-sar}). There are also currently no spatially consistent operational sea ice concentration product based on retrieval algorithms of SAR observations which are known to the author.

Forecasts can be used as predictors for the deep learning system since they provide information regarding how the conditions should evolve in the period after the forecast has been initialized. Thus giving the deep learning system insight into the future state of the domain while still facilitating operational usage by not relying on e.g. future observations. Hence, atmospheric variables from the AROME Arctic regional numerical weather prediction system will be included as input to the model. These variables (wind and temperature) have been chosen due to their physical impact on sea ice, and is assumed to encode information about the future state of sea ice concentration when seen in combination with past and present sea ice concentration by the deep learning system.

Finally, the highest resolution sea ice concentration product with an appropriate temporal frequency available are the sea ice charts produced by the NIS \citep{Dinessen2020}. The sea ice charts represent an interpretation of different sea ice observations delivered as a product directed towards operational users. Thus, the sea ice charts will serve as the ground truth for the model. Furthermore, as a deep learning system can increase its skill by combining correlated variables as input, this thesis will explore the impact caused by including several datasets covering both current observations, past trends as well as forecasted variables on different spatial resolutions as input predictors.

The following section will describe the domain covered for this thesis, followed by a rundown of the satellite products as well as physical models used. Table \ref{tab:data_overview} presents the different products used for this thesis, and whether the product is used to train or verify the model.

\begin{table}
    \caption{\label{tab:data_overview}List of the products used, their applications as well as temporal regime. The dashed line separates observational products (above) from model-based forecast products (below)}
    \centering
    \setlength{\arrayrulewidth}{0.5mm}
    \renewcommand{\arraystretch}{1.3}
    \begin{tabular}{lllll}
    \hline
    Product             & Variables           & Training & Verification & Time interval\\
    \hline
    Ice charts          & SIC                 & Yes      & Yes & Present / Future         \\
    OSI SAF SSMIS       & SIC trend           & Yes      & Yes & Past          \\
    OSI SAF CDR         & Ice edge length     & No       & Yes & Present         \\
    AMSR2               & SIC                 & No       & Yes & Future         \\
    \hdashline
    AROME Arctic        & t2m, X-wind, Y-wind & Yes      & No  & Future         \\
    NeXtSIM             & SIC                 & No       & Yes & Future         \\
    Barents-2.5         & SIC                 & No       & Yes & Future         \\
    \hline         
    \end{tabular}
\end{table}


\subsection{Region of interest}
\label{sec:roi}
The domain covered by the deep learning system, covers part of the European Arctic, with Svalbard off-centered. The coast of Northern-Norway is located on the Southern Border, the archipelago of Novaya Zemlya on the eastern border with Franz Josef Land located to its north. The northern border reaches ($88^\circ \text{N}, 79^\circ \text{E})$. The region is an intersection between the domain covered by the sea ice charts \citep{Dinessen2020} and AROME Arctic \citep{Mueller2017} as shown in Figure \ref{fig:studyarea}. The domain has a 1km spatial resolution, and contains $1792 \times 1792$ equidistant grid points. Compared to the AROME Arctic grid, the model domain has a reduced southern and eastern extent. The study area is a commercially active region, with regards to fishing, tourism and shipping \citep{Wagner2020}. Moreover, the sea ice area and extent in the domain have a strong seasonal variability \citep{Cavalieri2012}.

\begin{figure}
    \centering
    \includegraphics[width = 0.7\textwidth]{study_area.pdf}
    \caption{\label{fig:studyarea}The model domain is shown (dashed-contour) by sea ice concentration contours retrieved from a sea ice chart (15 Sep 2022). The colorbar is adapted from \protect\citet{WMO2014} to match the ice categories present in the sea ice charts.}
\end{figure}

The mean annual sea ice drift pattern governing the region of interest is in general positioned away from the Arctic Basin, with some local variations \citep{Barry1993}. In the area located towards the north of the domain, the sea ice drift is mainly driven by the Transpolar Drift Stream ocean current which transports the sea ice away from the Laptev Sea ($76^\circ$N, $125^\circ$E) and towards the Fram Strait located between Greenland and the Svalbard archipelago \citep{Colony1984}. Moreover, the sea ice drift pattern in the Fram Strait is characterized by the strongest drift for the entire Arctic Basin, and is positioned parallel to the coast of Greenland in a southwestward direction \citep{Barry1993}. Finally, a secondary drift pattern is observed originating in the Kara sea ($77^\circ$N, $77^\circ$E) during the winter, where sea ice is drifting towards the western coast of Svalbard between Franz Josef Land and Novaya Zemlya \citep{Kaur2018}.

\subsection{Observations}
Observations are used to convey the current state of sea ice concentration. There is a lack of consistent in situ observations of sea ice concentration, due to the remoteness of the region. Thus, most direct observations are ship-based concentration estimates \citep{Kern2019} or optical remote sensing, the latter is only available during daylight and cloud-free conditions. As a result, sea ice concentration is mainly observed automatically through passive microwave retrievals utilizing different sea ice retrieval algorithms e.g. \citep{Comiso1997,Spreen2008, Lavergne2019a}. Another source of sea ice observations are sea ice charts (\url{https://usicecenter.gov/}, Last Accessed 25 Jan 2023) or \citep{Dinessen2020}, which are manually drawn interpretations combining available sea ice concentration observations such as SAR, passive microwave and optical imagery. 

\subsubsection{Sea ice charts}
\label{sec:sea-ice-charts}
The sea ice charts utilized for this thesis are provided by the NIS. The product is manually drawn by a sea ice specialist, and is distributed every workday at 15:00 UTC. The sea ice specialist assesses available SAR scenes from Sentinel 1 and Radarsat 2. However, due to the spatial variability in daily SAR coverage (See Supporting Figure \ref{fig:S-sar}), visual, infrared and low resolution passive microwave observations are used as supplement to achieve a consistent spatial coverage \citep{Dinessen2020}. All observations used by the sea specialist are mainly gathered at the same date as the sea ice chart is drawn. The sea ice charts are not drawn onto any set resolution, although monitor zoom-level as well as pixels per inch and screen size of the used monitor are factors which determine the drawing resolution at any given time. Hence, a gridded representation of the ice charts is only a representation of the mean value of the polygons contained inside each grid cell. The sea ice charts used in this work has been interpolated onto a 1-km grid with the same projection as the AROME Arctic weather prediction system \citep{Mueller2017}.

With regards to consistency, it is noted that the current sea ice chart product have no easily identifiable way of noting which observations were used by the sea ice analyst to draw each segment of the chart. As the different satellite products used have different spatial scales, from meters to kilometers \citep{Dinessen2020}, the underlying uncertainty and ability to resolve structures varies both spatially and temporally. The published sea ice charts as seen in Figure \ref{fig:icechart} shows the available SAR coverage as black contours, which is the preferred data source for the ice analysts \citep{Dinessen2020}. 

\begin{figure}
    \centering
    \includegraphics[width=0.9\textwidth]{general_latest}
    \caption{\label{fig:icechart}Sea Ice chart produced by the NIS covering 10 Jan 2023 at 15:00 UTC. Sea ice concentration categories are drawn as filled contours. The black lines indicate the available SAR data used to draw the sea ice chart.}
\end{figure}

\begin{figure}
    \centering
    \includegraphics[width=0.8\textwidth]{2022-sic-distribution}
    \caption{\label{fig:2022-areadist-sic}Monthly distribution of each sea ice concentration category as respective fraction of the total mean sea ice concentration for the sea ice charts covering 2022.}
\end{figure}

Figure \ref{fig:2022-areadist-sic} shows the monthly distribution of sea ice concentration categories from the sea ice charts during 2022. As can be seen from Figure \ref{fig:2022-areadist-sic}, more than half of the region consists of ice free open water, with the other majority of an ice chart consisting of very close drift ice (concentration equal to or higher than 90\%). Moreover, Figure \ref{fig:2022-areadist-sic} shows the seasonal variability of the sea ice extent, with the ice free open water contributing between $\sim38$\% and $\sim75$\% of the entire domain depending on the month. The ice charts also resolve the intermediate sea ice categories, which for the current region is mostly related to the marginal ice zone (MIZ) and the ice edge.

\begin{figure}
    \centering
    \includegraphics[width = 0.8\textwidth]{autocorr_icechart}
    \caption{\label{fig:autocorr}Autocorrelation of the sea ice charts from 2022. The x-axis is the time lag between to entries 2022 sea ice chart timeseries. The y-axis is the normalized autocorrelation, i.e. autocorrelation at a certain time lag divided by the autocorrelation at time lag 0. The autocorrelation is computed for a period covering 31 days.}
\end{figure}

By inspecting Figure \ref{fig:autocorr}, it can be seen that the autocorrelation between two ice charts close in time is high. However, it can also be seen to steadily decline as the time lag increases. From the strong autocorrelation seen in Figure \ref{fig:autocorr}, it can be assumed that the persistence for short lead times (days) closely relates to the current sea ice concentration. Furthermore, the autocorrelation also renders previous sea ice concentration at short timescales as skillful at describing the current growth of the sea ice. The latter will be used as motivation to compute a linear sea ice concentration trend in Section \ref{sec:osisafssmis}.

The sea ice chart is an operational product mainly targeted for maritime operators. A single sea ice chart is usually drawn by one individual person from the NIS team. However, there are several sea ice specialists at the NIS whom draw ice charts. On the one hand, the operational nature of the product may influence the decision-making when creating the sea ice charts. Moreover, as a consequence of the human interaction with the production of the sea ice charts it can be assumed that there is an unknown degree of personal bias added to the data. However, a recent exercise conducted by the Danish Meteorological Institute compared the sea ice charts from five sea ice specialists covering a SAR scene of the Greenland sea to measure the inter specialist spread. The results from the exercise showed that the inter specialist spread was ($\sim9$ - 34) pixels on the computer screen, which transforms to a average ice edge displacement between the ice specialists of ($\sim380$ - $1540$) meters \citep{Kreiner2023}. On the other hand, the human involvement may also introduce a degree of quality control not seen in automatic sea ice concentration retrieval algorithms. Thus, the sea ice charts are assumed to have a low uncertainty, though there are no uncertainty estimates included \citep{Dinessen2020}.

In spite of the uncertainties outlined above, the sea ice charts are assumed to be the most accurate sea ice concentration product available for the purpose of high resolution data tailored towards operational end users. By utilizing the sea ice charts as the ground truth data when training the deep learning system, the developed model will fit towards the proposed high resolution operational use case.

\subsubsection{OSI SAF passive microwave sea ice concentration observations}
\label{sec:osisaf}
Two different sea ice concentration products are used from OSI SAF. OSI SAF Special Sensor Microwave Imager and Sounder (SSMIS) is an operational product delivering daily sea ice concentration on the northern (and southern) hemisphere. OSI SAF Climate Data Record (CDR) \citep{Soerensen2021} delivers sea ice concentration beginning in 1979 \citep{Lavergne2019}. The operational product will be used as a predictor for the model and for validation, whereas the CDR will be used only for validation purposes.

\subsubsection{OSI SAF SSMIS}
\label{sec:osisafssmis}
OSI SAF SSMIS is a passive microwave product derived from the SSMIS instrument. To convert brightness temperature to estimated sea ice concentration, a hybrid approach combining the Bootstrap algorithm \citep{Comiso1997} and the Bristol algorithm \citep{Smith1996} where the prior is used over open water and the latter used for ice concentrations above 40\% \citep{Tonboe2017}. The algorithm uses data from the 19GHz frequency channel (vertically polarized) and 37GHz channel (vertically and horizontally polarized), which are the two lowest spectral resolution channels for the SSMIS \cite{Tonboe2017}. Finally, atmospheric corrections are made using atmospheric analysis-data from the ECMWF. The end product is delivered every day on a 10km polar stereographic grid.

With regards to uncertainty, OSI SAF SSMIS is validated against Pan-Arctic sea ice charts from the U.S. National Ice Center as well as regional sea ice charts covering the Svalbard region from the NIS. Moreover, the operational product is required to have a bias and standard deviation less then 10\% ice concentration on an annual basis, when compared to the targets (\url{https://osisaf-hl.met.no/sea-ice-conc-edge-validation}, Last Accessed 24 Jan 2023) \citep{Lavelle2017}. This strengthens the assumption made at the end of Section \ref{sec:sea-ice-charts} regarding the accuracy of the sea ice charts and their validity in terms of serving as an independent source for reference.

In this work, the operational OSI SAF SSMIS dataset is used to compute a linear sea ice concentration trend in each grid cell. The trend is computed from a set amount of days backwards in time, and is on a 10 times coarser resolution with respect to the sea ice charts due to the resolution of the OSI SAF SSMIS grid. The motivation behind including a coarse resolution linear trend is to provide a high variance indicator of where the sea ice growth / decline has occurred.

The idea behind the computed trend is to encode multiple time-steps of sea ice concentration fields into a single two dimensional field, in line with the lack of temporal awareness of CNN-like architectures such as the U-Net. Moreover, the trend serves to limit the size of the training data, since the memory needed is equal to that of a single 2d-array regardless of the length of the trend. Furthermore, the linear sea ice concentration trend is computed from a product separate from the ice chart, with the intent to supply the model with correlated but not overlapping information since the current day ice chart is already used as a predictor. However, it should also be noted that the lack of sea ice charts during the weekends \citep{Dinessen2020} also limits the possibility of deriving a similar linear trend from the sea ice charts. As a sea ice concentration trend derived from the sea ice charts would be limited to at most five days, which is not the case for OSI SAF SSMIS where there are no temporal gaps in the dataset. The coarser resolution also contributes to the OSI SAF trend serving as complementary information to the ice charts, as the coarse resolution makes the trend less resolvent of the local variability which is seen in the ice charts.

The temporal length used when deriving the trend will have an impact on how the trend reflects the current growth and decline zones, especially with regards to the volatile position of the ice edge on a daily timescale but also due to the seasonal variability of the ice area \citep{Holland2016}. Hence, a too large lookbehind would cause a decorrelation between the current sea ice concentration and the computed trend. Nevertheless, Figure \ref{fig:autocorr} shows that there is significant autocorrelation for sea ice concentration on a short time-range, as described previously. However, a trend computed from a sufficiently long temporal window could be assumed to better represent the spatial distribution of seasonal sea ice concentration growth and decline rather than representing the current growth and decline.

\subsubsection{OSI SAF Climate Data Record}
\label{sec:osisafcdr}
As briefly mentioned in Section \ref{sec:introduction}, OSI SAF CDR combines observations from different sensors (SMMR, SSM/I, SSMIS) as well as numerical weather prediction fields from the ERA Interim reanalysis \citep{Dee2011}. The latter are utilized to correct for the atmospheric conditions. Two versions of the dataset has been used, version 2 (OSI-450) which covers (2011 - 2015), and the interim version (OSI-430-b) which cover (2016 - 2020) (\url{https://osisaf-hl.met.no/osi-450-430-b-desc})(Last Accessed 18 Jan 2023). Both products are processed using the same algorithms, ensuring consistency \citep{Lavergne2019a}. The interim version is serving as an extension of the original scope of OSI-450 (1979 - 2015), with a difference being its use of ECMWF analyses compared to ERA Interim reanalysis and different SSMIS input data (\url{https://osisaf-hl.met.no/osi-450-430-b-desc}, Last Accessed 24 Jan 2023). Regardless, both products will hereby be referred to in tandem as OSI SAF CDR

The OSI SAF retrieval algorithm has been shown to have strong correlation against ship based measurements \citep{Kern2019} as well as optical satellite observations during the summer \citep{Kern2020}. Hence, OSI SAF CDR is expected to serve as a low error representation of the Arctic sea ice concentration.

OSI SAF CDR is provided on a 25km spatial resolution with a Lambert Azimuthal Grid projection \citep{Soerensen2021}. The sea ice concentration data retrieved has been used to compute a climatological ice edge length for each day of the year, applying a daily mean across the time period (2011 - 2020). The ice edge length has been computed according to \citet{Melsom2019}, which will be described in Section \ref{sec:iceedgelength}. Note that though OSI SAF CDR provides a Pan-Arctic distribution of sea ice concentration, the data has been regridded onto the study region domain with the AROME Arctic projection and a 25km grid spacing before computing the ice edge length.

\begin{figure}
    \centering
    \includegraphics[width = 0.7\textwidth]{clim_iceedge.pdf}
    \caption{\label{fig:clim_iceedge} Seasonal variability of the climatological ice edge length computed from satellite observations during the period 2011 - 2020. Only the part of the field projected onto the region of interest has been considered.}
\end{figure}

As can be seen in Figure \ref{fig:clim_iceedge}, the sea ice edge in the study region experiences a strong seasonal variability. The computed climatological ice edge will be used as a normalization factor in order to use verification scores that are not seasonally dependent \citep{Goessling2016, Zampieri2019, Palerme2019}. Another benefit from utilizing a single ice edge length is to ensure that different sea ice products are normalized according to a common and independent factor.

\subsubsection{AMSR2}
\label{sec:amsr2}
The Advanced Microwave Scanning Radiometer 2 (AMSR2) data utilized for this thesis comes from the sea ice concentration product at the University of Bremen (\url{https://seaice.uni-bremen.de/sea-ice-concentration/amsre-amsr2/}) (Last Accessed 8 May 2023) \citep{Spreen2008}. AMSR2 is a passive sensor observing the microwaves emitted by the Earth, similar to the SSMIS sensor. AMSR2 is located on the JAXA GCOM-W1 satellite \citep{Melsheimer2019}, and the sea-ice concentration is retrieved using the ASI algorithm from \citet{Spreen2008}. The algorithm uses data from the 89GHz channel, which is the band with the highest spectral resolution, in both polarizations to determine the sea ice concentration. Bands at lower spectral resolutions are only used as weather filters, which can mask out false sea ice detected in the open ocean \citep{Spreen2008}. The resulting data is a pan-arctic sea ice coverage with a spatial resolution of 6.25km.

The current AMSR2 product was chosen as the ASI retrieval algorithm \citep{Spreen2008} results in a higher spatial resolution product compared to similar AMSR2 products such as the AMSR2 product from OSI SAF \citep{Lavelle2016}, which is delivered on a 10km spatial resolution.

\begin{figure}
    \centering
    \includegraphics[width=0.8\textwidth]{2022-sic-distribution_amsr2}
    \caption{\label{fig:dist-amsr2}Monthly distribution of each sea ice concentration class fraction for the AMSR2 dataset covering 2022. The data has been projected onto the region of interest.}
\end{figure}

Figure \ref{fig:dist-amsr2} shows the monthly distribution of sea ice contours for the AMSR2 dataset. Similarly to Figure \ref{fig:2022-areadist-sic}, a majority of the scenes are covered by Ice Free Open Water. However, Figure \ref{fig:dist-amsr2} shows the AMSR2 dataset has less very close drift ice compared to the sea ice charts, which may stem from an increased fast ice area and that the sea ice charts usually categorize all sea ice not part of the MIZ as very close drift ice (e.g. see Figure \ref{fig:icechart}). Furthermore, AMSR2 has a less resolved open water area compared to the sea-ice charts. This may be a result of AMSR2 being a algorithmically derived product, whereas the sea-ice charts are drawn for operational use such that regions of potential sea ice encounters are exaggerated to ensure maritime safety. Lastly, \citet{Spreen2008} demonstrated that the ASI algorithm provides appropriate results at concentrations above $65\%$ where the error does not exceed 10\% \citep{Spreen2008}. At 0\% sea ice concentration the algorithm has an error of 25\%, and for 100\% sea ice concentration the error is 5.7\%. The main contributor to the high uncertainties for sea ice concentration below 65\% are atmospheric interactions \citep{Spreen2008}.

Since the sea ice charts are treated as the ground truth during training of the deep learning model, it can be assumed that the model is best at predicting sea ice concentration distributions similar to those found in the training data. As such, the AMSR2 data will serve as an independent dataset with high spatial resolution, and will be used for validation only. Thus, the performance of the deep learning system can be inspected with regards to another dataset that is less similar than the sea ice charts, which measures the generalizability of the model.

\subsection{Forecasting systems} 
\subsubsection{AROME Arctic}
\label{sec:arome}
AROME Arctic is a non-hydrostatic, convection resolving high-resolution weather forecasting system which covers the European Arctic \citep{Mueller2017}. The model covers the European Arctic similarly to Figure \ref{fig:studyarea} which is the same domain though reduced, with a spatial resolution of 2.5km and 65 vertical levels. The first full year covered by AROME Arctic predictions was 2016, although the model have produced forecasts operationally since October 2015. AROME Arctic uses different data assimilation techniques for the atmosphere and surface variables. 3DVAR combines the atmospheric model background with observations and is forced by the deterministic ECMWF forecast (HRES), whereas optimal interpolation combines the surface model background with observations. Both parts of the data assimilation system is merged to produced the forecast analysis \citep{Mueller2017}. As previously mentioned, variables influencing the sea ice concentration can aid in improving the predictive capabilities of a deep learning system. While observational products described above such as the ice charts \citep{Dinessen2020} and OSI SAF SSMIS \citep{Tonboe2017} describe the condition and dynamics of the sea ice concentration. Integrating weather forecast data as part of the model input can be used to describe the interaction between sea ice and atmospheric variables, thus providing relevant variables for predicting sea ice concentration. For the scope of this thesis, 2-meter temperature as well as the u and v component of the 10-meter wind adjusted x and y dimensions of the region of interest has been selected.

Near surface winds influence the sea ice drift, with the sea ice in the European Arctic displaying a moderate to strong correlation between the sea ice drift speed and the wind speed during winter \citep{Spreen2011}. Moreover, sea ice drift speed is shown to be inversely proportional to the sea ice concentration \citep{Yu2020}. i.e. low concentration sea ice classes tend to have a higher drift speed than high concentration sea ice classes, although all classes display an increased drift speed given an increased near surface wind speed. Since the AROME Arctic grid is located at high latitudes, the direction of the u and v component of the 10-meter wind is transformed to conform to the x and y components of the region of interest.
Thus, including the x and y components (X-wind, Y-wind) of the near surface wind from AROME Arctic provides the deep learning system with a high resolution proxy for the predicted sea ice drift. 

Similarly, surface temperature influences the sea ice mass balance by melting or facilitating sea ice growth \citep{Hibler1979}, for example through the formation of melt ponds on top of the sea ice. The 2-meter temperature from AROME Arctic is intended to serve as a proxy for the sea ice growth, by including a spatial distribution of temperature as a predictor. This may be correlated to areas in the model domain experiencing mean positive (melt) or negative (growth) temperatures during the forecast period.

AROME Arctic is shown to have lower RMSE in both 2-meter temperature and 10-meter zonal wind speed than both the deterministic (HRES) and ensemble (ENS) forecasts from ECMWF, for all months when compared to measurements from 89 stations located in Finnmark, Svalbard as well as Jan Mayen and Bjørnøya \citep{Mueller2017}. Hence, it is reasonable to assume that extracting the wind and temperature fields from AROME Arctic will provide the most precise information with regards to the forecasted values and their spatial location, compared to global medium range numerical weather prediction systems such as the ECMWF Integrated Forecasting System (IFS) \citep{Haiden2022}. However, it is noted that operational numerical weather prediction systems such as those described by \citet{Mueller2017} and \citet{Haiden2022} are in constant development, with new improvements added without any retroactive effect for previous data. Firstly, the comparison made in \citet{Mueller2017} was with HRES and ENS as of Cycle 38r2 \citep{Bauer2013} which is not necessarily representative of the current state of both products. Secondly, significant advances in model development may cause data before and after the implementation date to be inconsistent, e.g. by introducing a permanent shift in bias for a variable. Problems regarding model updates could be avoided by using variables from a reanalysis product such as the Copernicus Arctic Regional Reanalysis (CARRA) \citep{Koeltzow2022}. However, CARRA similarly to other reanalysis products is produced in delayed-mode (see \url{https://climate.copernicus.eu/copernicus-arctic-regional-reanalysis-service}, Last Accessed 21 Jan 2023), which would inhibit the operational aspect of the developed deep learning system. It is also noted that CARRA specifically only have a 30 hour lead time, which limits the desired "up to 3 day" lead time for the developed deep learning system.

With regards to model development, a major development in AROME Arctic in terms of temperature representation over sea ice ocurred on 10 Oct 2018 (AROME Arctic Changelog, Last Access 21 Jan 2023), in the form of a \textit{snow on ice} variable. As this change is expected to have changed the distribution of 2-meter temperature significantly, especially over sea ice covered grid cells \citep{Batrak2019}, it has been opted to only consider near surface temperature data from AROME Arctic from 2019 and onwards. This decision is made to avoid having a shift in temperature distribution present in the data, which would exert a negative impact on training the deep learning system.

Though the different datasets in Table \ref{tab:data_overview} have been chosen with the intention to serve as independent products without any intra coupling, it is noted that the sea ice observations used to compute the sea ice concentration trend \citep{Tonboe2017} is also used to force AROME Arctic with sea ice concentration at the initial timestep \citep{Mueller2017}. It is suboptimal to provide input parameters which are derived from other input parameters, as their correlation may cause one of the input parameters to be rendered obsolete since the deep learning system may choose to only infer the redundant information from the other predictor. Nonetheless, it is assumed that the impact of the sea ice concentration forcing is low when combined with other surface forcings during the assimilation process. Furthermore, as the sea ice concentration is kept constant at all timesteps \citep{Mueller2017}, the correlation between sea ice concentration and atmospheric variables can be assumed to be decaying with time. Thus, both products will be used as input variables, and their overlap is assumed to be negligible.


\subsubsection{NeXtSIM}
\label{sec:nextsim}
The neXt generation Sea Ice Model (neXtSIM) is developed by the Nansen Environmental and Remote Sensing Center and performs the physical simulations for the neXtSIM-F deterministic forecasting platform \citep{Williams2021}. NeXtSIM-F assimilates sea ice concentration from operational OSI SAF sea ice concentration products \citep{Tonboe2017, Lavelle2016} and forces the model with oceanic and atmospheric forecasts. The data assimilation scheme computes the sea ice concentration analysis from a weighted average approach between the model background and the OSI SAF observations, and is applied with a daily frequency. Furthermore, the neXtSIM-F platform is not a coupled system, i.e. the neXtSIM sea ice model is not coupled to either and atmospheric or oceanic model. The version of neXtSIM-F data used for this thesis is supplied on a 3km polar stereographic grid on a pan-arctic domain. 

NeXtSIM differentiates itself from comparative physical sea ice models as it does not apply a rheology based on the Viscous-Plastic scheme. The rheology of a sea ice model refers to how the model relates ice deformation and ice thickness with the internal stresses in the ice \citep{Hibler1979}. Instead, NeXtSIM applies a brittle sea ice rheology, specifically the Brittle Bingham-Maxwell (BBM) rheology which treats the sea ice as a brittle material rather than a viscous fluid \citep{Olason2022}. With the implementation of a brittle rheology scheme, neXtSIM-F is the first sea ice forecasting system not to use a rheology from the viscous-plastic branch of rheologies \citep{Williams2021}.

With a forecast range of 7 days, data from neXtSIM-F will be used to validate the deep learning system against current high resolution operational sea ice forecasts by serving as a comparable product. NeXtSIM-F is the highest resolution model distributed as part of the Copernicus Marine Environmental Monitoring Service \citep{nextsimdata2020}.

\subsubsection{Barents-2.5}
\label{sec:barents}
Barents-2.5, is an operational coupled ocean and sea ice forecasting model under development at MET Norway \citep{Roehrs2022}. The model has been in operation since September 2021. Barents-2.5 poses the same resolution and projection as AROME Arctic, i.e. Lambert Conformal Conic with a 2.5km resolution \citep{Roehrs2022}. Furthermore, Barents-2.5 also forecasts with a lead time of up to 66 hours, which is the same as AROME Arctic. Since Barents-2.5 covers the same spatial domain as the deep learning system and forecasts with a lead time close to three days, its predicted sea ice concentration will be used for validation purposes.

The sea ice model used in Barents-2.5 is the Los Alamos sea ice model (CICE) version 5.1, which uses an Elastic Viscous Plastic sea ice Rheology \citep{Hunke2015}. Thus, the CICE model represents sea ice as a viscous fluid which creeps slowly given small stresses and deforms plastically under large stress. It is also noted that the elastic behavior was introduced to benefit the numerical aspects of the model, and can be considered unrealistic from a physical point of view \citep{Hunke1997}.

Barents-2.5 includes an Ensemble Prediction System with 6 members executed for each of the four model runs situated at (00, 06, 12 and 18) \citep{Roehrs2022}. As part of its forcing routine, Barents-2.5 performs non-homogenous atmospheric forcing of its ensemble members, with one member of each ensemble being forced with AROME Arctic while the rest of the members are forced using atmospheric data from ECMWF. As such, the members forced with AROME Arctic seem to perform best with regards to ocean currents, but the atmospheric forcing's impact on sea ice concentration performance is unknown at the time of writing (Johannes Röhrs, 2022, pers. commun.).

The data assimilation scheme applied for Barents-2.5 is a Deterministic Ensemble Kalman filter, which solves for the analysis with a background error covariance matrix estimated as the variance of the ensemble of background members \citep{Roehrs2022}. Furthermore, it has been expressed by the developers of Barents-2.5 that the model performance was unsatisfactory up until May / June 2022 due to spin up time of the data assimilation system (Johannes Röhrs, 2022, pers. commun.). As such, forecasts initiated prior to May 2022 will not be assessed for validation purposes due to the expected shift in performance as expressed by the model developers.

Similarly to the neXtSIM-F data in Section \ref{sec:nextsim}, Barents-2.5 will also be used to validate the deep learning system. However, the forecast range of Barents-2.5 is only 66 hours, which cuts it short of producing three full daily means. Furthermore, due to the ensemble setup of Barents-2.5, it is possible to present a forecast both through the ensemble mean as well as a pseudo deterministic run (single member). However, a forecast from a single Barents-2.5 member would still be influenced by the other ensemble members during the assimilation stage. 

ECMWF IFS is used to force both neXtSIM and Barents-2.5 with atmospheric variables, whereas the ocean and sea ice model TOPAZ \citep{Sakov2012} is used to force neXtSIM \citep{Williams2021} while only nudging the boundaries of Barents-2.5 \citep{Roehrs2022}. Moreover, both physical products are of a spatial and temporal scale for operational relevancy \citep{Wagner2020}, similar to the deep learning system. Thus given their differences in terms of ensemble setup, model coupling, sea ice rheology as well as domain coverage, both products have been included for validation of the deep learning system.

% Local bibliography
\biblio

\end{document}