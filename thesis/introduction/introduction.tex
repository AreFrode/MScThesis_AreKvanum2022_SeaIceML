\documentclass[../main/thesis.tex]{subfiles}

\begin{document}
\section{Introduction}
\label{sec:introduction}
The Arctic sea ice extent has continuously decreased since the first satellite observations of the Arctic was obtained in 1978 \citep{Serreze2019}, with an average decrease of 4\% per decade \citep{Cavalieri2012}. The summer months are experiencing the greatest loss of sea ice extent \citep{Comiso2017}, with models from the Coupled Model Intercomparison Project Phase 6 (CMIP6) projecting the first sea ice-free Arctic summer before 2050 \citep{Notz2020}. As a consequence of the sea ice retreat during the summer months, previously inaccessible oceanic areas have opened up causing an increase in maritime operations in the Arctic waters \citep{Ho2010, Eguiluz2016}. The expected influx of operators to the Arctic regions due to the prolonged open water season call for user-centric sea ice products on different spatial scales and resolutions to ensure maritime safety in the region \citep{Wagner2020, Veland2021}.

Current information on Arctic sea ice concentration can be discerned into several types of products with different spatial and temporal resolutions. Sea ice products designed for climate applications such as OSI-450, SICCI-25km and SICCI-50km provide daily sea ice concentration by merging observations from multiple sensors to create a historical dataset. The purpose of a climatology is to provide accurate reference data \citep{Lavergne2019} which can be used for e.g. forecast validation or anomaly detection. Satellite observations are also supplied as daily products, with a timeliness of a few hours on the same day and posing higher spatial resolutions than climatologies. For example, OSI-401-b \citep{Tonboe2017} and OSI-408 \citep{Lavelle2016} provide single sensor daily averaged sea ice concentration covering the northern and southern hemisphere, and can be used to force numerical weather prediction systems which only resolve the atmosphere \citep{Mueller2017}.

Sea ice models are physically based models resolving the growth and movement of sea ice forward in time. Standalone models such as CiCE \citep{Hunke1997} and neXtSIM \citep{Williams2021} can be used independently or coupled with ocean models \citep{Roehrs2022} to create sea ice forecasting systems for short lead times. Finally, sea ice charts drawn analogously by a sea ice specialist merge recent sea ice observations from different sensors and satellites into a single daily product. The Ice Service of the Norwegian Meteorological Institute (NIS) provides regional ice charts covering the European Arctic. The product consists of polygons which are drawn to match the current resolution of the available observations, which range from 50m to several kilometers, and are assumed to have a low uncertainty due to the quality control exerted by the sea ice specialist \citep{Dinessen2020}.

The previously mentioned sea ice products serve different use cases, and it is possible to infer a correlation between the spatial and temporal resolution of a product and its application scenario for maritime end users. While lower resolution products at larger temporal time scales can be used in long term planning, regional high resolution products delivered at a high frequency can assist strategic decision making and short term route planning \citep{Wagner2020}. However, it is currently reported by end users that available operational passive microwave satellite products are of a too low resolution, partly due to their insufficient ability to resolve leads and other high-resolution information necessary for maritime safety. Moreover, it is also reported that sea ice forecasting systems lack desired verification, are inadequate for operational use as well as being difficult to integrate with a vessel where computational resources and data-bandwidth are limited \citep{Veland2021}. Though sea ice charts provide maritime end users in the Arctic with information regarding where sea ice has been observed in the time after the previous ice chart has been published, the ice charts does not provide a description on the future outlook. Thus, the responsibility of interpreting the ice charts and other available sea ice information with a outlook on future development is delegated to the end-user and relies on their experience to ensure a continued safe navigation \citep{Veland2021}.

As such, a different approach to short-range sea ice forecasting may be necessary to deliver short-term sea ice information on a spatial scale that is relevant for end-users. Thus, this thesis proposes an alternative forecasting scheme that applies Convolutional deep learning in the form of a modified U-Net architecture \citep{Ronneberger2015} to deliver a short lead time (1 - 3 days), 1km resolution forecasting product over a subsection of the European Arctic by utilizing the aforementioned Ice Charts as the ground truth. Moreover, the product is verified with regards to the position of the ice edge, which aims to demonstrate the operational relevance of the product \citep{Veland2021, Melsom2019}. 

There have been made previous attempts to develop deep learning sea ice forecasting systems. \citet{Andersson2021} propose IceNet, a pan-arctic covering U-NET which predicts monthly averaged sea ice concentration (SIC) with 6 month lead time at a 25 km spatial resolution \citep{Andersson2021}. The model classifies sea ice concentration into one of the three classes open-water, marginal ice or full ice. IceNet showed an overall improvement over the numerical SEAS5 seasonal forecasting system \citep{Johnson2019} for 2 months lead time and more, with the greatest improvement seen in the late summer months. The model is trained on SIC data provided by the European Organization for the Exploitation of Meteorological Satellites (EUMETSAT) Ocean and Sea Ice Satellite Application Facilities (OSI SAF) dataset \citep{Lavergne2019}, as well as other climate variables obtained from the ERA5 reanalysis \citep{Hersbach2020}. Their model was validated against SEAS5, which is a seasonal forecasting system developed by the European Center for Medium-Range Weather Forecasts (ECMWF) \citep{Johnson2019}.

Similarly, \citet{Liu2021} propose a Convolutional long short-term memory network (ConvLSTM) which forecasts SIC with a lead time up to 6 weeks. The model uses climate variables and SIC from two reanalysis products ERA-Interim \citep{Dee2011} and ORAS4 \citep{Balmaseda2013}, covering the Barents Sea with a domain size of 24 (latitude) x 56 (longitude). Their results showed skill in beating numerical models as well as persistence. 

Models such as those noted above consider input variables obtained from climatologies, and represent SIC on spatial scales far larger than what is needed for an operational short-term sea ice forecast. The possibility of using higher resolution input data was explored by \citet{Fritzner2020}, which combined OSISAF SIC, sea surface temperature from the Multi-scale Ultra-high Resolution product, 2 meter air temperature from the ERA5 reanalysis as well as SIC from sea ice charts produced by the NIS. Fritzner et.al. developed a Fully Convolutional Network (FCN), which achieved similar performance to the Metroms coupled ocean and sea ice model version 0.3 \citep{Kristensen2017}. However, due to computational constraints of training the FCN, the subdomain was reduced to a resolution of 224 x 224 pixels which translates to 10 - 20km \citep{Fritzner2020}. Thus, the product has a limited accuracy for short term operational usage, similar to \citep{Andersson2021} and \citep{Liu2021}.

Contrary to the authors above, \citet{Grigoryev2022} propose a 10 day lead time regional forecasting system with a 5km spatial resolution trained on a sequential (traditional) and recurrent U-Net architecture. The authors used 5km AMSR-2 sea ice concentration as the ground truth variable, and regrid atmospheric variables from the NCEP Global Forecast System (\url{https://www.emc.ncep.noaa.gov/emc/pages/numerical_forecast_systems/gfs.php}) to match the resolution of the ground truth. Their results showed that the recurrent setup slightly outperformed the sequential architecture for predictions with a lead time up to 3 days, with both architectures significantly outperforming persistence and the linear trend. However, the sequential architecture tended to outperform the recurrent architecture for 10 day forecasts, as the recurrent model was trained without weather data as it only had a lead time of 3 days.

As mentioned in \citep{Andersson2021, Fritzner2020}, the computational cost of producing a forecast using a pre-trained model is low, such that a laptop running consumer hardware is able to generate a forecast in seconds or minutes depending on the availability of a Graphics Processing Units (GPU). This is in stark contrast to numerical sea ice models, which could run for several hours on high-performance systems \citep{Andersson2021}. Training a model is a one time expense, and can be efficiently performed on a GPU. With the increased complexity, efficiency and availability of high end computing power, smart usage of the available memory allows for model training using high resolution fields. Current GPUs have seen a significant increase in the available video memory, which allows for higher resolution data to be utilized during training. This work will exploit the recent advances in GPU development, as well as incorporating techniques to reduce the floating point precision of the input meteorological variables, circumventing a reduction of the spatial resolution as seen in previous works.

Moreover, the U-Net architecture is part of the supervised learning paradigm of machine learning, which require labelled samples in order to train the network \citep{Ronneberger2015}. Furthermore, U-Nets perform pixel-level prediction where each pixel is classified according to a category. This work will utilize the image-to-image predictive capabilities of the U-Net to create a semantic segmentation based on its input variables simulating a forward in time propagation of the sea ice concentration akin to a physical model. This allows for the inspection of how changes to the architecture as well as input data configurations affect the behavior of the forecasting system.

In the present work, the development of a deep learning forecasting system will be explored. The choice and tuning of hyperparameters will be reasoned in light of the physical processes affecting sea ice and the surrounding variables. Furthermore, the quality of the machine learning forecasting system will be assessed against relevant benchmarks such as persistence, physical models and linear regression of the observed sea ice concentration. Due to the operational nature of the developed forecasting product, ice edge aware validation metrics such as the Integrated Ice Edge Error \citep{Goessling2016} will be central to the performance analysis. Furthermore, this thesis aims at providing the framework for which a future operational sea ice prediction system can be built upon. As such, the choice and structure of data will be made with a potential operational transition in mind.

This thesis aims at exploring the following research questions: 
\begin{itemize}
    \item Can a deep learning system resolve regional sea ice concentration for high resolution, short lead time forecasts? 
    \item How does a high resolution, short lead time U-Net forecasting system resolve the translation and accumulation of sea ice compared to a physical based model
    \item  In what sense can a deep learning model be explainable / made transparent to explain the statistical reasoning behind the physical decision-making
\end{itemize}

% Her må vi inn og reference sections når de eksisterer :)))))
The thesis is structured as follows. The first section will describe the datasets used, followed by the second section which will do a rundown of the methodological framework necessary to develop the U-Net as well as validation metrics used to assess forecast skill. The third section will detail the development process behind the U-Net, with the fourth section exploring the physical connections of the model. The fifth section will detail the performance assessment of the forecasts. In the sixth section, a discussion of the findings will be conducted, with the seventh and final section presenting conclusions and future outlook.

\biblio

\end{document}